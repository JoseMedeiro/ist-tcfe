\section{Introduction}
\label{sec:introduction}

% state the learning objective 
The aim of this laboratory assignment is to design and analyse an AC/DC converter that would transform an input AC voltage of amplitude 230V and frequency of 50 Hz to an output DC voltage of amplitude 12V and frequency 50Hz, using an Envelope Detector, a Voltage Regulator and a Transformer. The final circuit is shown below

\begin{figure}[h] \centering
\includegraphics[height=7cm]{Circuit_base.pdf}
\caption{Circuit analysed.}
\label{fig:Circuit_Base}
\end{figure}

even though the circuit used in the calculations and simulations is the following:

\begin{figure}[h] \centering
\includegraphics[height=7cm]{Circuit_base_simple.pdf}
\caption{Simplified circuit analysed.}
\label{fig:Circuit_Base_S}
\end{figure}

The quality of the circuit, when compared to the others, is determined by calculating the merit of the work: 

\begin{equation}
	Merit = \frac{1}{cost*(ripple_{reg}+|average_{reg}-12|+10^{-6})}
\label{eq:i1}
\end{equation}

The cost of the components are the following: cost of resistors = 1 monetary unit (MU) per kOhm, cost of capacitors = 1 MU/uF and cost of diodes = 0.1 MU per diode.

\newpage

The ratio of the transformer and the values of the resistors, the capacitor and number of diodes are presented in the following table.

\begin{table}[h]
  \centering
  \begin{tabular}{|l|r|}
    \hline    
    {\bf Name} & {\bf Value} \\ \hline
    \input{../mat/VALUES_tab}
  \end{tabular}
  \caption{Variables preceded by \# is of type {\em resistance} and expressed in Ohm; variables preceded by \& is of type {\em capacitance} and expressed in Farad; A variable preceded by £ is of type {\it cost} and expressed in Unit of Cost; $n$ is adimentional.}
  \label{tab:INTRO_VALUES}
\end{table}

In Section~\ref{sec:simulation}, the circuit is analysed by simulation using a default model fo the diode in NGSpice. The converter was simulated for 10 periods and the voltage average and ripple were measured using built in functions, and plots of the output of the Envelope Detector and the Voltage Regulator Circuits are presented.\par
In Section~\ref{sec:analysis}, a theoretical analysis of the circuit is presented, using theoretical models of the diodes to predict the output of the Envelope Detector and the Voltage Regulator Circuits. The output DC level and the voltage ripple are calculated as well as the plots for the output deviation and the output of the regulator.\par
The conclusions of the assigment are outlined in  Section~\ref{sec:conclusion}, which also includes a comparison of the results obtained.