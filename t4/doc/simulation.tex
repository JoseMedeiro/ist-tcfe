\section{Simulation Analysis }
\label{sec:simulation}

\subsection{Audio Amplifier converter graphs}

In this section we evaluated the Audio Amplifier proposed.
In order to analise the circuit after the transiant period, we have chosen $t \in \left[ 1 ms; 20 ms \right[ $.

The graphs for NGSpice are displayed here, alongside the table with the required elements:

\begin{figure}[h] 
	\centering
	\vspace{-3cm}
	\includegraphics[height=11cm]{../sim/trans.pdf}
	\caption{$v_{in}$ and $v_{out}$.}
	\label{fig:SIM_TRANS}
\end{figure}

It is worth noting that the output is not distorted, making this amplifier a viable option for usage.

\begin{figure}[h] \centering
	\vspace{-3cm}
	\includegraphics[height=11cm]{../sim/amp.pdf}
	\caption{$ \left | v_{out}/v_{in} \right |$.}
\end{figure}

\newpage

\begin{figure}[h] \centering
	\vspace{-3cm}
	\includegraphics[height=11cm]{../sim/ampdb.pdf}
	\caption{$db(v_{out})$ and $max(db(v_{out}))-3$.}
\end{figure}

\begin{figure}[h] \centering
	\vspace{-3cm}
	\includegraphics[height=11cm]{../sim/phdb.pdf}
	\caption{Phasor of $v_{out}$, rad}
\end{figure}

\newpage

Table~\ref{tab:SIM_DC} shows the simulated operating point results for the circuit under analysis. From here we can see that the transistors stay in the FAR region.
%Colocar a tabela 
\begin{table}[h]
  \centering
  \begin{tabular}{|l|r|}
    \hline    
    {\bf Name} & {\bf Value} \\ \hline
    \input{../sim/SIM_DC_tab}
  \end{tabular}
  \caption{Variables are of type {\it voltage} and expressed in Volt.}
  \label{tab:SIM_DC}
\end{table}

Table~\ref{tab:SIM_RES} shows the simulated frequency response results for the circuit.
\begin{table}[h]
  \centering
  \begin{tabular}{|l|r|}
    \hline    
    {\bf Name} & {\bf Value} \\ \hline
    \input{../sim/SIM_RESULTS_tab}
  \end{tabular}
  \caption{Merit is in {\it per voltage per cost} and expressed in Volt$^{-1}$UC$^{-1}$; $f1 - f2$ is of the type {\it frequency} and expressed in Hz; other variables are adimentional.}
  \label{tab:SIM_RES}
\end{table}

Table~\ref{tab:SIM_ZIN} and table~\ref{tab:SIM_ZOUT} show the input and output impedances, respectivly.
%Colocar a tabela 

\begin{table}[h]
\centering
\begin{minipage}[t]{0.50\linewidth}
 	 \begin{tabular}[t]{|l|r|}
    	\hline    
   		{\bf Name} & {\bf Value} \\ \hline
    	\input{../sim/SIM_ZIN_tab}
  	\end{tabular}
  	\caption{ Variables are of type {\it impedance} and expressed in Ohm.}
  	\label{tab:SIM_ZIN}
\end{minipage}

\begin{minipage}[t]{0.50\linewidth}
  	\begin{tabular}[t]{|l|r|}
    	\hline    
   		{\bf Name} & {\bf Value [A or V]} \\ \hline
    	\input{../sim/SIM_ZOUT_tab}
  	\end{tabular}
	\caption{ Variables are of type {\it impedance} and expressed in Ohm.}
  	\label{tab:SIM_ZOUT}
\end{minipage}
\end{table}

