\section{Introduction}
\label{sec:introduction}

% state the learning objective 
The aim of this laboratory assignment was to create an audio amplifier circuit and analyze it both theoretically and by simulation using Ngspice, designing the gain and output stages . It receives an audio input of 10mV as is connected to an 8 Ohm speaker. The source has an 100 Ohm impedance and the circuit is supplied with 12V by a DC source (Vcc).

In order to have a good voltage amplifier, a high gain $A_v$, a hight input impedance $Z_i$ and a low output impedance are required, so that the input and output voltages do not get degraded. There are two stages in the audio amplifier circuit: the gain and output stages. 

Firstly, in the gain stage a common emitter amplifier with degeneration was used as it leads to a high $Z_i$ as well as a high gain $A_v$. However, it also has high $Z_o$, that degrades the output signal. The next stages then, solves this problem. For this stage a NPN transistor was used. 

Secondly, in the output stage a common collector amplifier was used because it not only keeps the hight gain $A_v$ from the first stage but also has a low $Z_o$. This happens because of its high input impedance which connects to the lower output impedance (but still high) from the previous stage and the gain in this section is approximately 1. For this stage a PNP transistor was used.

The quality of the audio amplifier, comparing to others, is determined by a merit classification system. It takes into account the cost of the components used, the voltage gain, the cut off frequency and the bandwidth. The merit is calculated according the following equation:

\begin{equation}
	MERIT = \frac {Voltage Gain * Bandwidth}{Cost * Lower Cut Off Frequency}
	\label{eq:i1}
\end{equation}

And the cost of the components is: resistors = 1 monetary unit (MU) per kOhm; capacitors = 1 MU/uF; transistors = 0.1 MU per diode.

Figure~\ref{fig:Circuit_Base} shows the layout of the implemented circuit and its actual architecture implemented for each stage will sooner be analyzed in detail.

\begin{figure}[h] 
	\centering
	\includegraphics[height=7cm]{Circuit_base.pdf}
	\caption{Circuit analysed.}
	\label{fig:Circuit_Base}
\end{figure}

\newpage

Table~\ref{tab:VALUES} shows the resistors, capacitors and number of transistors, as well as the total cost of the circuit.

\begin{table}[h]
  \centering
  \begin{tabular}{|l|r|}
    \hline    
    {\bf Name} & {\bf Value} \\ \hline
    \input{../mat/Values_tab}
  \end{tabular}
  \caption{Variables preceded by \# is of type {\em resistance} and expressed in Ohm; variables preceded by \& is of type {\em capacitance} and expressed in Farad; A variable preceded by £ is of type {\it cost} and expressed in Unit of Cost; the numbers of $BJTs$ are in units.}
  \label{tab:VALUES}
\end{table}

In Section~\ref{sec:simulation}, the circuit is analyzed by simulation in Ngspice, using the Philips BJT’S model BC557A (PNP) and BC547A (NPN) for the transistors. The simulation allowed to compute the input and output frequencies (where the difference between both corresponds to the bandwidth), as well as the gain.	\par
Then, in Section~\ref{sec:analysis}, a theoretical analysis of the circuit in presented, using a suitable OP, making sure the transistors are in the Forward Active Region) and incremental theoretical models studied in order to predict the gain and the input and output impedances for each stage. \par
In the conclusion, Section~\ref{sec:conclusion} the theoretical and the simulation results were compared.