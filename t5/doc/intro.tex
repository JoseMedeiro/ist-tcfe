\section{Introduction}
\label{sec:introduction}

% state the learning objective 
The aim of the laboratory assignment was to create a BandPass Filter (BPF, using an OP-AMP, shown in Figure 1. It should have a central frequency of 1 kHz a gain of 40 dB, or 100 times, at this frequency. In order to implement the BPF, the following components were available:

\begin{itemize}
	\item One 741 OP-AMP;
    \item At most three $1k \Omega$ resistors;
    \item At most three $10k \Omega$ resistors;
    \item At most three $100k \Omega$ resistors;
    \item At most three $220nF$ capacitors;
    \item At most three $10 \mu$ capacitors.
\end{itemize}


In order to have a good voltage amplifier, a high gain $A_v$, a hight input impedance $Z_i$ and a low output impedance are required, so that the input and output voltages do not get degraded. There are two stages in the audio amplifier circuit: the gain and output stages. 

The quality of the BPF, comparing to others, is determined by a merit classification system. It takes into account the cost of the components used, the voltage gain, the cut off frequency and the bandwidth. The merit is calculated according the following equation:

\begin{equation}
	MERIT = \frac {1}{Cost * (Gain deviation + Central Frequency deviation + 10{-6} )}
	\label{eq:i1}
\end{equation}

And the cost of the components is: 

\begin{itemize}
	\item cost of resistors = 1 monetary unit (MU) per $k \Omega$;
    \item cost of capacitors = 1 MU/ $\mu F$;
    \item cost of transistors = 0.1 MU per transistor.
\end{itemize}

Figure~\ref{fig:Circuit_Base} shows the layout of the implemented circuit.

\begin{figure}[h]
	\centering
	\includegraphics[height=7cm]{Circuit_base.pdf}
	\caption{Circuit analysed.}
	\label{fig:Circuit_Base}
\end{figure}

\newpage

Table~\ref{tab:VALUES} shows the resistors, capacitors and number of transistors, as well as the total cost of the circuit.

\begin{table}[h]
  \centering
  \begin{tabular}{|l|r|}
    \hline    
    {\bf Name} & {\bf Value} \\ \hline
    \input{../mat/Values_tab}
  \end{tabular}
  \caption{Variables preceded by \# is of type {\em resistance} and expressed in Ohm; variables preceded by \& is of type {\em capacitance} and expressed in Farad; A variable preceded by £ is of type {\it cost} and expressed in Unit of Cost; the numbers of OP-AMPs are in units.}
  \label{tab:VALUES}
\end{table}

In Section~\ref{sec:simulation}, the circuit is analyzed by simulation in Ngspice, using the OPAMP model provided. It allowed us to obtain the output voltage gain in the passband, the central frequency and the input and output impedances.	
\par
Then, in Section~\ref{sec:analysis}, the circuit is analyzed theoretically using Octave tools, where the frequency response $V_o(f)/V_i(f)$, the  gain and the input and output impedances at the central frequency are determined.
\par
In the conclusion, Section~\ref{sec:conclusion} the theoretical and the simulation results were compared.