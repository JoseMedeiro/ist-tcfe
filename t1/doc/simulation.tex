\section{Simulation Analysis }
\label{sec:simulation}

\subsection{Operating Point Analysis}

Table~\ref{tab:op} shows the simulated operating point results for the circuit under analysis. All the currents were measured from the lower numbered node to the higher one; so, the direction of the current that passes through $R_6$ is, if the result is positive, from $v_4$ to $v_7$.

\begin{table}[h]
  \centering
  \begin{tabular}{|l|r|}
    \hline    
    {\bf Name} & {\bf Value [A or V]} \\ \hline
    \input{../sim/op_tab}
  \end{tabular}
  \caption{Operating point. A variable preceded by @ is of type {\em current}
    and expressed in Ampere; other variables are of type {\it voltage} and expressed in
    Volt.}
  \label{tab:op}
\end{table}

It is important to notice that it has been necessary to implement an extra test voltage sorce $V_3$ connecting $v_4$ to $R_6$ providing 0V so that is does not interfere with the rest of the circuit and enable us to measure the current $I_c$ flowing in the above mentioned resistor. Therefore, this creation does not change any results and it is only based on the Ngspice requirements to define a current controlled voltage source.

Compared to the theoretical analysis the simulation showed practically identical results, except for a small divergence in the last decimal place that probably occurs when NGSpice rounds the numbers. Thus, the maximum relative error is $10^{-5}$.
It is worth mentioning that ngspice software also uses the same mathematical methods as octave to find results, hence, this result was already expected.









