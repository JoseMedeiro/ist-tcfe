\section{Conclusion}
\label{sec:conclusion}

To conclude, in this laboratory assigment the objective of analysing a circuit containing a sinusoidal voltage source and a capacitor has been achieved. 
Static, time and frequency analyses have been performed both theoretically using the Octave maths tool and by circuit simulation using the Ngspice tool.

\begin{table}[h]
\centering
\begin{minipage}[t]{0.33\linewidth}
 	 \begin{tabular}[t]{|l|r|}
 	   \hline    
 	   {\bf Name} & {\bf Value [A or V]} \\ \hline
 	   \input{../mat/PASSO1_tab}
 	 \end{tabular}
 	 \label{tab:PASSO1_CONCLUSAO}
\end{minipage}
\begin{minipage}[t]{0.33\linewidth}
  		\begin{tabular}[t]{|l|r|}
    	\hline    
   		{\bf Name} & {\bf Value [A or V]} \\ \hline
    	\input{../sim/PASSO1_tab}
  		\end{tabular}
  	\label{tab:SIM_PASSO1_CONCLUSAO}
\end{minipage}
  	\caption{Operating point for $t<0$. in Octave and NGSpice, respectivily. A variable preceded by @ is of type {\em current} are expressed in Ampere; other variables are of type {\it voltage} and expressed in Volt. (As shown in Tables \ref{tab:TEO_PASSO1} and \ref{tab:SIM_PASSO1})}
\end{table}

\newpage

\begin{table}[h]
\centering
\begin{minipage}[t]{0.33\linewidth}
 	 \begin{tabular}[t]{|l|r|}
 	   \hline    
 	   {\bf Name} & {\bf Value [A or V]} \\ \hline
 	   \input{../mat/PASSO2_tab}
 	 \end{tabular}
 	 \label{tab:PASSO2_CONCLUSAO}
\end{minipage}
\begin{minipage}[t]{0.33\linewidth}
  		\begin{tabular}[t]{|l|r|}
    	\hline    
   		{\bf Name} & {\bf Value [A or V]} \\ \hline
    	\input{../sim/PASSO2_tab}
  		\end{tabular}
  	\label{tab:SIM_PASSO2_CONCLUSAO}
\end{minipage}
  \caption{Operating point for $t=0$. in Octave and NGSpice, respectivily A variable preceded by @ is of type {\em current} and expressed in Ampere; variables preceded by * are expressed in Farad Ohm; variables preceded by \# is of type {\em resistance} and expressed in Ohm; other variables are of type {\em voltage} and expressed in Volt. (As shown in Tables \ref{tab:TEO_PASSO2} and \ref{tab:SIM_PASSO2})}
\end{table}

The results match considerably, as expected, since this is a relatively straightforward circuit containing only one capacitor apart from linear components, therefore the theoretical and simulation models shouldn't differ. Values of the order of $10^{-14}$ and lower should be taken as~0.